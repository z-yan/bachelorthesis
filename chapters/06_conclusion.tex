%!TEX root = ../thesis.tex

\chapter{Conclusion and Outlook}
\label{cha:conclusion}
\section{Conclusion}
This chapter concludes this bachelor thesis. The thesis detailed the development of a dedicated graphical web front end for a word-embedding-enabled relational database system. The lack of a suitable graphical front end for exploratory use cases featuring a word-embedding-enabled relational database system was identified as an issue. The fundamentals of word embedding techniques, their real-world applications and previous work on integrating their features into a relational database management system were examined. An extensive analysis of the problem in hand followed, from which the requirements for a web front end were derived. Based on the requirements analysis, a three-tier application architecture was chosen for the web application. The front end was then implemented using state-of-the-art web technologies and frameworks and thoroughly tested. Finally, an evaluation of the finished software product showed its suitability for the intended use cases, its conformance with modern user interface design principles and its adequate performance. \\ \\
Several conclusions could be drawn from this thesis:
\begin{itemize}
	\item Using a three-tier architecture consisting of several components communicating with each other via messages makes sense for a database system front end and provides high flexibility and modularity.
	\item When designing a graphical user interface for a web application, trade-offs between the interface's usability and its exposing of the application's functionality should sometimes be made. Even so, UX design principles should always be taken into consideration in the process of creating a graphical user interface.
	\item Even within the highly-specific context of a word-embedding-enabled database system, adaptation and code reuse of existing external library components remain an important part of web application development. Using state-of-the-art web frameworks both on the front-end and the back-end sides of the application is key to its swift development.
	\item The execution time of the complex vector operations within a word-embedding-enabled database system is the biggest bottleneck when interacting with it via a graphical front end. Moreover, its performance is highly dependent on available physical resources.
\end{itemize}

\section{Outlook}
Despite the web application's suitableness for the use cases described in Chapter \ref{cha:usecase} and its fulfillment of the derived requirements, a few unresolved issues remain. The web application also lacks some desirable functionalities.

First of all, several changes to the web application's front end should be made. Even though responsive web frameworks were used in the application's development and its graphical interface was tested on several different screen sizes and resolutions, it is not yet fully compatible with mobile devices such as smartphones or tablets. Thus, it cannot be considered truly responsive. A code revision of the HTML templates and adjustments to the CSS stylesheets would be of benefit for allowing the comfortable usage of the web application on all kinds of devices. Secondly, taking the web application's accessibility to visually impaired people and people with color vision deficiency into consideration was out of the scope of this thesis. Extensive changes to the user interface should therefore be made in order to ensure its accessibility to people with different kinds of impairments, which may require the implementation of dedicated components to replace some of the external libraries used or modifications to them.

From an implementational point of view, the web application's code could benefit from a refactoring for the purpose of better component modularization. More precisely, it could make better use of AngularJS features such as multiple controllers, modules, services and factories. This would result in the code being more easily maintainable and more extensible.

During this thesis' writing, the unsatisfying performance of some word-embedding-enabled queries was identified as a major issue and a hindrance to the user's smooth interaction with the front end. The web application's back end could therefore be extended by advanced caching functionalities in order to improve the response time of the most frequently used queries.

Furthermore, several security issues were identified during the application's development. SQL injection and execution of denial of service attacks are both possible. Theoretically, the web front end could be exploited as an attack vector, as query customizability was chosen over implementing advanced query parsing to protect against malicious SQL queries. Both the front end and the back end's query and request processing could be improved by a dedicated query parser in order to protect the database system against attacks executed via the front end.

There are several possibilities for adding new functionalities to the web application. For example, an entirely new view focused on word embeddings data sets could be implemented. This could allow the comparison of the results of different word embedding operations using word embeddings trained on various corpora, thus providing further linguistic insights. Another possible feature is a functionality of executing batch word embedding operations on a column of a previous query's results. This is another use case which could benefit from the implementation of an advanced word-embedding-enabled query parser.