%!TEX root = ../thesis.tex

\chapter{Introduction}
\pagenumbering{arabic}
\section{Motivation}
We live in the Age of Big Data. A blog post by Bernard Marr published in May 2018 cites intriguing statistics about the quantity of data created by mankind which could only confirm this statement. Over 90 percent of the data in the world was generated over the last two years alone. Currently, 2.5 quintillion bytes of data are created each day, and this rate is expected to accelerate. Furthermore, every minute 16 million text messages and 156 million emails are sent \cite{forbes-data}. 

Extracting relevant information from such a huge volume of data proved itself a complex, time- and resource-consuming task. A large part of the business-relevant information in data is contained in so-called \textit{unstructured data} \cite{clarabridge}, that is, information that is not organized in any predefined manner, primarily in the form of text, such as the aforementioned text messages and emails. Developing special techniques to find patterns in unstructured data and interpret it has been a major focus of computer science research in the last few decades, as efficient processing of texts using computers turned out to be particularly challenging. Examples for areas in which such techniques are being developed are \textit{data mining}, \textit{natural language processing} and \textit{text analytics}.

Among these, natural language processing is especially relevant to this thesis. Natural language processing is defined as "\textit{a branch of artificial intelligence that deals with analyzing, understanding and generating the languages that humans use naturally in order to interface with computers in both written and spoken contexts using natural human languages instead of computer languages}" \cite{nlp-def}. One of the approaches in the field of natural language processing is \textit{word embedding}, a method which maps words or phrases to vectors of real numbers, allowing the efficient processing of enormous amounts of textual data by computers. Its applications include accurate machine-aided translation \cite{zou2013bilingual}, sentiment analysis \cite{hamilton2016inducing} and phrase auto-completion \cite{Sordoni:2015:HRE:2806416.2806493}.

Despite the rise of other database models, relational database management systems (RDBMS) still dominate the database market. As of May 2018, four out of the top five most popular database management systems are relational ones \cite{db-engines}. Using the possibilities for information retrieval provided by word embedding could provide new insights into the semantic content of existing and new relational databases. In order to achieve this, systems extending queries on relational databases with new functions that use operations on word embeddings have to be developed. The development of one such system based on \textit{PostgreSQL\footnote{\url{https://www.postgresql.org/}}}, called \textit{FREDDY}\footnote{\url{https://wwwdb.inf.tu-dresden.de/research-projects/freddy/}} (\textbf{F}ast Wo\textbf{r}d \textbf{E}mbed\textbf{d}ings in \textbf{D}atabase S\textbf{y}stems), was started in 2017 at the Technische Universität Dresden by Michael Günther. It \textit{"is able to use word embeddings [to] exhibit the rich information encoded in textual values"} \cite{freddy-github} by extending SQL queries with operations on word embeddings.

\section{Goal}
The developed system proved itself to be effective and resource-efficient. However, it suffered from the lack of an intuitive, user-friendly interface. Interacting with FREDDY required the use of traditional PostgreSQL tools, such as the terminal-based front end \textit{psql}. Such tools are unfitting for  use cases such as investigating the system's capabilities and analyzing its performance. A dedicated user interface for FREDDY has to be both well-suited for use cases such as these and allow the intuitive adjustment of specific parameters used in the system's algorithms. The goal of this bachelor thesis project is to develop a graphical user interface for FREDDY demonstrating its features in the form of a web application. The finished \textit{demo application} has to fulfill several objectives:
\begin{enumerate}
	\item it has to allow the user of the application to test the system's features by \textit{exploring several different example datasets} and \textit{running word-embedding-enabled SQL queries} on them;
	\item the web application also has to offer the functionality of \textit{visualizing FREDDY's word embeddings operations' performance} in real time;
	\item it should provide a transparent way to \textit{adjust options and parameters} of the database management system powering it;
	\item the demo application is supposed to use \textit{modern web technologies and frameworks} and to conform to \textit{contemporary user interface design principles} in order to ensure a smooth user experience.
\end{enumerate}

This bachelor thesis describes in detail the development of the aforementioned web application.

\section{Structure}
The thesis follows a bottom-up structure, starting with the research and technologies based on which the demo application was built, proceeding to analyze its related core requirements, then detailing the architecture of its implementation, and finally providing a walkthrough of the finished software product's interface and an evaluation of the fulfillment of its requirements.

In the next chapter of this thesis (Chapter \ref{cha:fundamentals}), I introduce the fundamental natural language processing and database concepts and technologies that are key to the demo application's purpose. Chapter \ref{cha:usecase} proceeds to describe the use cases for the developed application and an example scenario of a user's interaction with it and derives a list of functional and non-functional requirements from them. In Chapter \ref{cha:implementation}, I detail the architecture of the software solution and its implementation specifics, including used technologies, programming languages, frameworks and external libraries. This chapter also contains a description of the data sets used in the application's deployment. Chapter \ref{cha:evaluation} provides a walkthrough of the finished application's graphical user interface. Furthermore, I assess the quality of its user interface according to contemporary user experience (UX) design principles and evaluate the other non-functional requirements for it. Chapter \ref{cha:conclusion} examines the possibilities for the further development of the demo application and concludes this bachelor thesis.